\documentclass[compress,]{beamer}

%presentation layout

\mode<presentation>
{
  \usetheme{Berlin}
  % \usecolortheme{dove}
  \setbeamercolor{structure}{bg=white,fg=black}
  \setbeamercolor{normal text}{bg=white,fg=black}
  \setbeamercolor{titlepage}{bg=white,fg=black}
  \setbeamercolor{titlelike}{bg=white,fg=black}
  \setbeamercolor{palette primary}{bg=white}
  \setbeamercolor{palette secondary}{bg=white, fg=white}
  \setbeamercolor{palette tertiary}{bg=white, fg=white}
  \setbeamercolor{palette quarternary}{bg=black}
  \setbeamercovered{transparent}
  \useinnertheme{rectangles}
  %\usefonttheme{serif}
}

\setbeamertemplate{navigation symbols}{}

%loading packages
\usepackage[ngerman]{babel}
\usepackage[T1]{fontenc}
\usepackage[utf8]{inputenc}
\usepackage{graphicx}
\usepackage{amsmath}

% vorgeplaenkel
\title[ZäPFchen-Einführung]{ZäPFchen-Einführung}

\author{Ständiger Ausschuss aller Physik-Fachschaften}

\institute[Zusammenkunft aller Physik-Fachschaften]

\date{11. November 2021\\Karneval}

\subject{ZäPFchen-Einführung}

\begin{document}

\begin{frame}
  \titlepage

  % \begin{figure}
  %   \centering
  %   \includegraphics[]{}
  % \end{figure}
\end{frame}

\section{ZäPFchen-Einführung}

%%%Folie 1


\section{Was passiert mit mir?}

\subsection{Ablauf}

\begin{frame}{Ablauf}

  \begin{figure}
    \centering
    \includegraphics[width=0.6\textwidth]{Zeitplan_Göttingen.jpeg}

   % \caption{Viel Spaß, Essen und Unterhaltung. Dazwischen etwas Arbeit in AK...}
  \end{figure}

\end{frame}

%%%Folie 3

\subsection{Die ZaPF, Göttingen und ich}
\begin{frame}{Die ZaPF, Göttingen und ich}
  \begin{itemize}[<+->]
  \item Erster Schritt geschafft. Ihr seid hier!
  \item \textbf{Digitale Welt von Göttingen erklären}
  \item Köln ist ein Stadtteil von Göttingen. Obviously
  \end{itemize}

\end{frame}


\subsection{Aufbau}

\begin{frame}{Plenum? Plenums? Plena?}

  \begin{itemize}[<+->]
  \item \textbf{Was ist es?} Das oberste beschlussfähige Gremium der ZaPF.
  \item \textbf{Wer?} Das ZaPF-Plenum sind wir alle gemeinsam.
  \item \textbf{Wann?} Anfang und Ende.
  \item \textbf{Wie?} Geschäftsordnung steht im Tagungsheft
  \item \textbf{Was passiert?} Wahlen, Abstimmungen, Vorstellen von Ergebnissen

    Ergebnisse kommen aus Arbeitskreisen, die im Anfangsplenum eingeteilt werden.
  \item Beim halbdigitalen Plenum werden einige Abstimmungen digital stattfinden. Einige Personenwahlen werden im Anschluss an die ZaPF per Briefwahl stattfinden.
  \end{itemize}

%%%Folien 4

\end{frame}

\begin{frame}{Was ist ein Arbeitskreis?}

  \begin{itemize}[<+->]
  \item \textbf{Wieso?} Im Plenum diskutieren ist anstrengend.

    In kleine(re)n Gruppen zu sprechen ist effektiv und zeitsparend
  \item \textbf{Wann?} Zweistündige Slots an allen Tagen außer am Ersten und Letzten.
  \item \textbf{Wie?} 5 bis 30 Teilnehmende: Hybrid

    Teilweise mit Gästen, Vorträge, Workshops oder andere Formate

  \item Arten?
    \begin{itemize}[<+->]
    \item \glqq{}Normal\grqq{}: Diskussionen und Austausch
    \item Austausch-AK: viele kurze Themen
    \item Folge-AK: braucht meist Vorwissen
    \item Spaß-AKs: Bier-Austausch, Fachschaftsfreundschaften
    \end{itemize}
  \end{itemize}

\end{frame}

%%%Folie 5

\subsection{ZaPF-Wiki}

\begin{frame}{Hey¸ hey, Wiki! Hey, Wiki, hey!}

\begin{itemize}[<+->]
	\item ZaPF-Wiki = Arbeitsplattform
	\item Sammelt Ergebnisse von ZaPFen
	\item Bereitet ZaPFen vor
	\item Im Wiki könnt ihr euch mit euren ZaPF Account anmelden
\end{itemize}

\end{frame}


%\subsection{ZaPF- Forum}

%\begin{frame}{Das Forum}

%\begin{itemize}[<+->]
	%\item ZaPF-Forum = Austauschplattform zu einzelnen Themen und AKs
	%\item ZaPF übergreifende Plattform
	%\item Auch hier könnt ihr euch mit euren ZaPF Account anmelden

%\end{itemize}

%\end{frame}

\begin{frame}{Messengers}

\begin{itemize}[<+->]
	\item Schreibe mit ZaPFika live und in Farbe
	\item Gruppen sind toll und seeeehr viele
	\item In folgenden Medien gibt es ZaPF Gruppen:
		\begin{itemize}
			\item Telegram
			\item Signal
		\end{itemize}
	\item Die Einladungslinks stehen im Tagungsheft. Hoffentlich!
	\item Und hier auf BBB in den geteilten Notizen
\end{itemize}

\end{frame}


%\begin{frame}{Benutzerkonto erstellen}
%%  \centering
%%  \Huge{Alt + Tab}
%\begin{figure}
%\centering
%\includegraphics[scale=0.15]{ZaPFWiki_2.pdf}
%
%\caption{https://zapf.wiki/Spezial:Benutzerkonto\_beantragen}
%\end{figure}
%
%\end{frame}

%%%Folie 6


\subsection{Protokolle}

\begin{frame}{Protokolle}

  \begin{itemize}[<+->]
  \item Zu jedem AK \emph{muss} ein Protokoll geschrieben werden.
  \item \textbf{Wie?} Vorlagen sind im Wiki.

  \item Speichert diese auch im Wiki ab!
  \item \textbf{Warum?}
    \begin{itemize}
    \item Weitere Arbeit an selben Themen (Folge-AKs)
    \item Fasst AKe zusammen
    \item Füllen Gedächtnislücken
    \end{itemize}
  \item Werden für die Nachwelt im Reader und im Wiki erhalten
  \end{itemize}

\end{frame}


%%% Folie 7

\subsection{Kleinigkeiten}

\begin{frame}{Kleinigkeiten}

\begin{itemize}[<+->]
	\item Ja wir LIEBEN Abkürzungen! Im Tagungsheft findet ihr Näheres
	\item Enten sind toll
	\item Kuschel AKs sind kuschelig
	\begin{itemize}
		\item spontan, jederzeit und überall
		\item Kuschel AK rufen und kuscheln, die wollen
		\item Wichtig: Niemand muss!
	\end{itemize}
	\begin{figure}
		\centering
		\includegraphics[width=0.3\textwidth]{Ente.pdf}
	\end{figure}
	
\end{itemize}
\end{frame}
%%% Folie 8

\section{Viel Spaß}

\begin{frame}{Viel Spaß}

Viel Spaß, fleißiges Arbeiten und viel Spaß!

\end{frame}

\end{document}

% Folgender Abschnitt nur für Emacs inklusive Auctex
%%% Local Variables: 
%%% mode: latex
%%% TeX-master: t
%%% End: