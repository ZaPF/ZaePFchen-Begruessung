\documentclass[compress,]{beamer}

%presentation layout

\mode<presentation>
{
  \usetheme{Berlin}
  % \usecolortheme{dove}
  \setbeamercolor{structure}{bg=white,fg=black}
  \setbeamercolor{normal text}{bg=white,fg=black}
  \setbeamercolor{titlepage}{bg=white,fg=black}
  \setbeamercolor{titlelike}{bg=white,fg=black}
  \setbeamercolor{palette primary}{bg=white}
  \setbeamercolor{palette secondary}{bg=white, fg=white}
  \setbeamercolor{palette tertiary}{bg=white, fg=white}
  \setbeamercolor{palette quarternary}{bg=black}
  \setbeamercovered{transparent}
  \useinnertheme{rectangles}
  %\usefonttheme{serif}
}

\setbeamertemplate{navigation symbols}{}

%loading packages
\usepackage[ngerman]{babel}
\usepackage[T1]{fontenc}
\usepackage[utf8]{inputenc}
\usepackage{graphicx}
\usepackage{amsmath}

% vorgeplaenkel
\title[ZäPFchen-Einführung]{ZäPFchen-Einführung}

\author{Ständiger Ausschuss aller Physikfachschaften}

\institute[Zusammenkunft aller Physikfachschaften]

\date{30. Mai 2018}

\subject{ZäPFchen-Einführung}

\begin{document}

\begin{frame}
  \titlepage

  % \begin{figure}
  %   \centering
  %   \includegraphics[]{}
  % \end{figure}
\end{frame}

\section{ZäPFchen-Einführung}

%%%Folie 1


\section{Was passiert mit mir?}

\subsection{Ablauf}

\begin{frame}{Ablauf}

  \begin{figure}
    \centering
    \includegraphics[scale=0.25]{WiSe18_Programm.pdf}

    \caption{Viel Spaß, Exkursionen, Essen und Unterhaltung. Dazwischen etwas Arbeit in AKs...}
  \end{figure}

\end{frame}

%%%Folie 3

\subsection{Aufbau}

\begin{frame}{Plenum? Plenums? Plena?}

  \begin{itemize}[<+->]
  \item \textbf{Was ist es?} Das oberste beschlussfähige Gremium der ZaPF.
  \item \textbf{Wer?} Das ZaPF-Plenum sind wir alle gemeinsam.
  \item \textbf{Wann?} Anfang und Ende.
  \item \textbf{Wie?} Geschäftsordnung steht im Tagungsheft
  \item \textbf{Was passiert?} Wahlen, Abstimmungen, Vorstellen von Ergebnissen

    Ergebnisse kommen aus Arbeitskreisen, die im Anfangsplenum eingeteilt werden.
  \end{itemize}

%%%Folien 4

\end{frame}

\begin{frame}{Was ist ein Arbeitskreis?}

  \begin{itemize}[<+->]
  \item \textbf{Wieso?} Im Plenum diskutieren ist anstrengend.

    In kleine(re)n Gruppen zu sprechen ist effektiv und zeitsparend
  \item \textbf{Wann?} Zweistündige Slots an allen Tagen außer am ersten und letzten.
  \item \textbf{Wie?} 5 bis 30 Teilnehmer

    Teilweise mit Gästen, Vorträge, Workshops oder andere Formate
  \item Arten?
    \begin{itemize}[<+->]
    \item ``Normal'': Diskussionen und Austausch
    \item Austausch-AK: viele kurze Themen
    \item Folge-AK: braucht meist Vorwissen
    \item Spaß-AKs (Ausnahmen): Bier-Austausch, Fachschaftsfreundschaften
    \end{itemize}
  \end{itemize}

\end{frame}

%%%Folie 5

\subsection{ZaPF-Wiki}

\begin{frame}{Hey¸ hey, Wiki! hey, Wiki, hey!}

\begin{itemize}[<+->]
	\item ZaPF-Wiki = Arbeitsplattform
	\item Sammelt Ergebnisse von ZaPFen
	\item Bereitet ZaPFen vor
\end{itemize}

\end{frame}

\begin{frame}{Benutzerkonto erstellen}
%  \centering
%  \Huge{Alt + Tab}
\begin{figure}
\centering
\includegraphics[scale=0.15]{ZaPFWiki_2.pdf}

\caption{https://zapf.wiki/Spezial:Benutzerkonto\_beantragen}
\end{figure}

\end{frame}

%%%Folie 6


\subsection{Protokolle}

\begin{frame}{Protokolle}

  \begin{itemize}
  \item<1-> Zu jedem AK \emph{muss} ein Protokoll geschrieben werden.
  \item<2-> \textbf{Wie?} Vorlagen sind im Wiki.

  \item<3-> Speichert diese auch im Wiki ab!
  \item<3-> \textbf{Warum?}
    \begin{itemize}
    \item<4-> Weitere Arbeit an selben Themen (Folge-AKs)
    \item<5-> Fasst AKe zusammen
    \item<6-> Füllen Gedächtnislücken
    \end{itemize}
  \item<7-> Werden für die Nachwelt im Reader und im Wiki erhalten
  \end{itemize}

\end{frame}


%%% Folie 7

\subsection{Kleinigkeiten}

\begin{frame}{Kleinigkeiten}

\begin{itemize}
	\item<1-> Ja wir LIEBEN Abkürzungen, im Tagungsheft findet ihr näheres
	\item<2-> Enten sind toll
	\begin{figure}
		\centering
		\includegraphics[scale=0.1]{Ente.pdf}
	\end{figure}
	
\end{itemize}
\end{frame}
%%% Folie 8

\section{Viel Spaß}

\begin{frame}{Viel Spaß}

Viel Spaß und fleißiges arbeiten!

\end{frame}

\end{document}

% Folgender Abschnitt nur für Emacs inklusive Auctex
%%% Local Variables: 
%%% mode: latex
%%% TeX-master: t
%%% End:
